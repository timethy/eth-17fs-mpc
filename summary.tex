\documentclass[landscape,a4paper,8pt]{scrartcl}

\usepackage{layout}
\usepackage{changepage}   % for the adjustwidth environment
%\usepackage{extsizes}
%\usepackage{savetrees}

%------------------------------------
%  FOOTER SETTINGS
%------------------------------------
\pagestyle{fancy}
\renewcommand{\headrulewidth}{0pt}
\fancyhead{}
\renewcommand{\footrulewidth}{0pt}
\fancyfoot[L]{Tim Taubner (taubnert@ethz.ch)}
\fancyfoot[C]{Model Predictive Control - \today}
\fancyfoot[RO, LE] {\thepage}

\newcommand{\mc}[1]{\mathcal{#1}}

\DeclareMathOperator\rank{rank}
\DeclareMathOperator\dom{dom}

%------------------------------------
%  BEGIN DOCUMENT
%------------------------------------

\begin{document}
%\sffamily

%--CONTENT--
\begin{multicols*}{3}
\section{Systems Theory}
\section{Unconstrained Control}

\section{(Convex) Optimization}

\paragraph{General Problem}
$\min_{x \in \dom(f)} f(x)$ s.\ t.\ $g_i(x) \leq 0$ and $h_j(x) = 0.$
%for $i = 1,\dots,m, j = 1,\dots,p$.

\paragraph{RHC}
\paragraph{QP with substitution}
\paragraph{QP with out substitution}

\subsection{Duality}
\paragraph{Lagrangian Dual Function}
\begin{align*}
	L(x,\lambda,\nu) & = f(x) + \sum_{i=1}^{m}\lambda_i g_i(x) + \sum_{i=1}^{p}\nu_i h_i(x) \\
	d(\lambda,\nu) & = \inf_{x \in \mc{X}} L(x,\lambda,\nu) \quad \text{ i.e. } \nabla_x L(x, \lambda, \nu) = 0
%	&= \inf_{x \in \mc{X}} \left[ f_0(x)+\sum_{i=1}^{m}\lambda_i f_i(x)+\sum_{i=1}^{p}\nu_i h_i(x)\right]
\end{align*}
\paragraph{Dual Problem (always convex)} 
$\max_{\lambda,\nu} d(\lambda,\nu)$ s.\ t.\ $\lambda \geq 0$. \\
Optimal value is lower bound for primal: $d^* \leq p^*$.

If primal convex, \emph{Slater condition} (strict feasibility) implies \emph{strong duality}:
\begin{align*}
	\left\{x \left| \right. Ax=b, f_i(x)<0, \right\} \neq \emptyset \Rightarrow d^*  = p^*
\end{align*}

\paragraph{Karush-Kuhn-Tucker (KKT) Conditions}
are necessary for optimality (and sufficient if primal convex).
\begin{itemize}
	\item Primal Feasibility:
		\begin{align*}
			f_i(x^*) &\leq 0 \quad i=1,\dots,m\\
			h_i(x^*) &=0 \quad i=1,\dots,p
		\end{align*}
	\item Dual Feasibility:  $\lambda^* \geq 0$
	\item Complementary Slackness:
		\begin{align*}
			\lambda_i^* \cdot f_i(x^*) = 0 \quad \quad i=1,\dots,m
		\end{align*}
	\item Stationarity:
		\begin{align*}
			&\nabla_x L(x^*,\lambda^*,\nu^*) =0 \\
			%&\nabla f_0(x^*) + \sum_{i=1}^m\lambda_i^*\nabla f_i(x^*)+  \sum_{i=1}^p\nu_i^*\nabla h_i(x^*)
		\end{align*}
\end{itemize}

\subsection{Constrained Finite Time Optimal Control (CFTOC)}

\subsection{Invariance}
\subsection{Feasability, Stability}

\subsection{Practical MPC}
\subsection{Robust MPC}
\paragraph{Tube-MPC}

\subsection{Explicit MPC}

\subsection{Hybrid MPC}

\section{Numerical Optimization}
Gradient, Newton, Interior Point


\section{Observer Based Control}
\subsection{LTI Observer}
LTI System: 
\begin{align*}
& x(k) = A x(k-1) + B u (k-1) + v(k-1) \\
& z(k) = H x(k) + w(k)
\end{align*}

Linear Static Gain Observer (Luenberger Observer):
\begin{align*}
&\hat{x}(k) = A \hat{x}(k-1) + B u (k-1) + K(z(k) - \hat{z}(k)) \\
&\hat{z}(k) = H ( A \hat{x}(k-1) + B u(k-1)) \\
&e(k)= (I-K \, H)\, A \,  e(k-1)
\end{align*}
$e(k) \to 0$ for $k \to \infty$ if and only if $(I-K \, H)\, A$ is stable.\\

Steady State: 
\begin{align*}
&\hat{x}(k) = (I-K_\infty H )\, A\, \hat{x}(k-1) + (I-K_\infty) B \, u(k-1) + K_\infty z(k)
\end{align*}
The steady-state KF is one way to design the observer gain $K$ (optimal in minimizing the Steady State mean squared error). \\

%Observer Estimation error:
%\begin{align*}
%e(k)= (I-K \, H)\, A \,  e(k-1)
%\end{align*}



%\begin{tabular}{ll}
%LTI system: & $x(k) = A x(k-1) + B u (k-1) + v(k-1)$ \\
%& $z(k) = H x(k) + w(k)$
%\end{tabular} 
%
%\begin{tabular}{ll}
%Observer: & $\hat{x}(k) = A \hat{x}(k-1) + B u (k-1) + K(z(k) - \hat{z}(k))$ \\
%& $\hat{z}(k) = H ( A \hat{x}(k-1) + B u(k-1))$
%\end{tabular}
%
%\begin{tabular}{ll}
%Steady State: & $\hat{x}(k) = (I-K_\infty H )\, A\, \hat{x}(k-1) + ... $ \\
% & $(I-K_\infty) B \, u(k-1) + K_\infty z(k)$ \\
%Estim. error: & $e(k)= (I-K \, H)\, A \,  e(k-1)$
%\end{tabular}

$(A,H)$ detectable $ \Rightarrow$ $K$ exists such that $(I-K \, H) A$ is stable.

\subsection{Static State Feedback Control}
Design of a controller without paying attention to the state estimation:
\begin{align*}
x(k) &= A x(k-1) + B u(k-1) \tag{Process without noise} \\
z(k) &= x(k)  \tag{Perfect State information} \\
u(k) &= F \cdot z(k) = F \cdot x(k) \tag{Control Law}
\end{align*}
%Control law: $u(k) = F \cdot z(k) = F \cdot x(k)$ (needs to be designed) \\

Closed loop dynamics: $x(k) = (A+B F)$. Hence system is stable if $(A+B F)$ is stable. Such an $F$ exists only if $(A,B)$ is stabilizable. \\
If $(A,B)$ is stabilizable and $(A,G)$ detectable, then $F$ is given by
\begin{align*}
F = -(B^T P B + \bar{R})^{-1} \cdot B^T P A; \qquad P\geq 0
\end{align*}
$P$ from DARE: $P = A^T P A + \bar{Q} - A^T P B( B^T P B + \bar{R})^{-1} \cdot B^T P A$

%\subsection{Linear Static Gain Observer (Luenberger Observer)}
%\begin{align*}
%\hat{x}(k) &= A \hat{x}(k-1) + B \, u(k-1) + K(z(k) - \hat{z}(k)) \\
%\hat{z}(k) &= H(A \hat{x}(k-1) + B u(k-1)) \\
%u(k) &= F \, \hat{x}(k)
%\end{align*}


\subsection{Separation Principle (Linear Systems only)}
Combining Luenberger Observer and Static State Feedback control yields:
\begin{align*}
\begin{bmatrix}
x(k) \\ e(k) 
\end{bmatrix}
= 
\begin{bmatrix}
A + BF & - B F \\ 0 & (I-KH) A
\end{bmatrix}
\cdot
\begin{bmatrix}
x(k-1) \\ e(k-1)
\end{bmatrix}
\end{align*}
Eigenvalues of closed loop are given bei Eigenvalues of $(I-KH) A$ and $(A + BF)$. System is stable as long as there exists no $|\lambda| \geq 1$. \\

%Note:
%\begin{itemize}
%\item Including noise does not affect stability.
%\item Separation principle generally does not hold for nonlinear systems.
%\end{itemize}


\subsection{Separation Theorem}
%Given a LTI System with $v(k-1) \sim \mathcal{N}(0,Q)$ and $w(k-1) \sim \mathcal{N}(0,R)$

\begin{enumerate}
\item Design steady-state KF which does not depend on $\bar{Q}, \bar{R}$. $\Rightarrow \hat{x}(k)$
\item Design state-feedback $u(k) = F x(k)$ and put both together.
\end{enumerate}

\end{multicols*}
\end{document}
