\documentclass[10pt,parskip]{scrartcl}
\usepackage{calc}
\usepackage[landscape,paper=a4paper,left=3mm, right=3mm,top=0mm, bottom = 3mm,includehead,headheight=5mm,headsep=2mm]{geometry}
\usepackage{multicol}
%Fr angepasste Aufzhlungen ohne Abstnde
\usepackage[alwaysadjust]{paralist}


%++++++++++++++++++++++++++++++++++++++ MEINE SACHEN
\usepackage[ngerman]{babel} % Unterstuetzung fuer deutsche Trennung und
%   zB Datum-Formatierung

%\usepackage{cmbright} % CMBRIGHT für alles! aus Formeln , weglassen für  Computer Modern und Roman Math 
\usepackage[utf8]{inputenc} % Einfache Eingabe von Umlauten

\usepackage[T1]{fontenc}

\usepackage{multirow} % Zeilenzusammen nehmen
%\usepackage{lpic} % text über grafiken

\renewcommand\familydefault{\rmdefault}
\usepackage[]{amsmath}   % diese zwei Pakete werden hier nicht
\usepackage{amssymb}   % gebraucht sind aber sehr sinnvoll
\usepackage{amsfonts}
\usepackage{empheq}
\usepackage{color}
\usepackage{graphicx}
%\usepackage[derived]{SIunits} %Units
\graphicspath{/}
%\usepackage{listings}  keine Ahung lol

 %PLATZSPAAREN SPACING 
\setlength{\parskip}{0pt}
\setlength{\parsep}{0pt}


\usepackage[]{titlesec} % SPACING überall definieren
\titlespacing*{\section}{0pt}{0pt}{5pt} %indent,space befor, space after
\titlespacing*{\subsection}{0pt}{0.3cm}{0pt}
\titlespacing*{\subsubsection}{0pt}{0pt}{0pt}
\titleformat*{\section}{\fontencoding{OT1}\fontfamily{cmr} \fontseries{bx}\fontshape{sc} \fontsize{15pt}{15pt} \selectfont} %italic
\titleformat*{\subsection}{\fontencoding{OT1}\fontfamily{cmr} \fontseries{bx}\fontshape{sc} \fontsize{12pt}{12pt} \selectfont} %italic
\def\thesection{}
\def\thesubsection{\arabic{subsection}}

\include{macros} %Macros reinladen
\include{def}


%\setlength\headsep{9pt}
\setlength\columnseprule{0.5pt}


%

%\setlength{\abovedisplayskip}{0pt}
%\setlength{\abovedisplayshortskip}{2pt}
%\setlength{\belowdisplayskip}{5pt}
%\setlength{\belowdisplayshortskip}{2pt}

%

%%Header
\usepackage{fancyhdr}
\pagestyle{fancy}
\rhead{Florian Petit, Philippe Petit}
\lhead{\thepage}
\chead{Summary Recursive Filtering and Estimation}
%Befehl zum Gliedern des Textes in Spalten%

%++++++++++++++++++++++++++++++++++++++++++++++++++++++++++++++++++++++++
\newcommand{\E}{\textrm{E}}
\newcommand{\V}{\textrm{Var}}

\begin{document}
\begin{multicols*}{3}





\subsection{Linear Systems}

Solution to linear ODE $\dot x(t) = A^c x(t) + B^c u(t), x0:=x(t_0)$\\
$x(t) = e^{A^c(t-t_0)} x_0 + \int_{t_0}^t e^{A^c(t-\tau)} B^c u(\tau) d \tau$

Discretization of LTI c-t ss model\\
$t_0 = t_k, t = t_{k+1}, t_{k+1}-t_k = T_s, u(t) = u(t_k) \forall t \in [t_k, t_{k+1})$\\
$x(t_{k+1}) = e^{A^c T_s} x(t_k) + \int_{0}^{T_s} e^{A^c(T_s-\tau')} B^c d \tau' u(t_k)$\\
$x(t_{k+1}) = A x(t_k) + B u(t_k)$

Coordinate Trafo may facilitate system analysis\\
$\tilde x = Tx, det(T) \ne 0;$ where $T = [e_1, ..., e_n]^{-1}$\\
EW: det$(A - \lambda I)=0,$ EV: $(A-\lambda_i I) e_i = 0$


\underline{Asymptotic stability} (stays bounded and returns to 0): \\
$\lim_{k  \to \infty} x(k) = 0, \forall x(0) \in \mathbb{R}$ with $u = 0$\\
Necessary \& sufficient cond. on the eigenvalues:
$\vert \lambda_i \vert < 1, \forall i = 1,...,n$ \\
Proof by trafo:\\
$\tilde x(k+1) = T A T^{-1} \tilde x(k) = diag( \lambda_n) \tilde x(k)  = \Lambda \tilde x(k)   $\\
State $\tilde x(k)$ can be expressed as function of $\tilde x(0) = T x(0)$: \\
$\tilde x(k) = diag( \lambda_n^k) \tilde x(0)  = \Lambda^k \tilde x(o)$   \\
$\tilde x(k) =\Lambda^k \tilde x(0) \Rightarrow \vert \tilde x_i(k) \vert \leq \vert \lambda_i \vert^k  \vert \tilde x_i(0) \vert$


\underline{Global Lyapunov Stability} (only sufficient)\\
$x=0$ is GAS if there is a function $V: \mathbb{R}^n \rightarrow \mathbb{R}$:\\
$\|x| \rightarrow \infty \Rightarrow V(x) \rightarrow  \infty$  \\
$V(0) = 0 \ and \ V(x)>0, \forall x \ne 0 $ \\
$V(f(x(k))) - V(x(k)) < 0, \forall x \ne 0 $ 

Lyapunov Stability for linear system $x(k+1)=A x(k)$ \\
$V(Ax(k)) - V(x(k)) = x^T(k)(A^T P A -P)x(k) < 0, P>0$ \\
\underline{Lyapunov equation} \\
\hspace*{10mm}$A^T P A -P = -Q, Q > 0 $\\
Fulfilled if $A$ has all eigenvalues inside the unit circle

Infinite horizon cost-to-go for asymptotically stable system $x(k+1)=A x(k)$\\
$\Psi(x(0)) =  \sum_{k=0}^\infty x(k)^T Q x(k) = \sum_{k=0}^\infty x(0)^T (A^k)^T Q A^k x(0) = x(0)^T P x(0) $ \\
$P = \sum_{k=0}^\infty  (A^k)^T Q A^k = Q + \sum_{k=1}^\infty (A^k)^T Q A^k =$\\
$ Q + A^T (\sum_{k=0}^\infty (A^k)^T Q A^k) A =  Q+ A^T P A $

\underline{Controllability}: goal state $x^*$ can be reached in finite time: max in $n$ steps \\
$x^* = A^N x(0) + (B \ AB \dots A^{n-1}B)
\begin{pmatrix}
 u(N-1) \\
 u(N-2) \\
 \vdots \\
 u(0) 
\end{pmatrix}$\\
$= A^N x(0) + \mathcal C U$\\
Necessary and sufficient cond: rank$(\mathcal C)=n$. \\
Stabilizability: input sequence exists, that returns the state to the origin asymptotically: A system is stabilizable iff all its uncontrollable modes are stable\\
if rank$([\lambda_i I-A \vert B]) = n \ \forall \lambda_i \in \Lambda_A^+ \Rightarrow (A,B)$ is stabilizable\\
$\Lambda_A^+$ is the set of eigenvalues of A lying on or outside the unit circle.


\underline{Observable}: measurements $y(0), y(1),...,y(N-1)$ uniquely distinguish $x(0)$

$\begin{pmatrix}
y(0)\\
y(1)\\
\vdots\\
y((N-1)
\end{pmatrix}
=
\begin{pmatrix}
C\\
CA\\
\vdots\\
CA^{N-1}
\end{pmatrix}
x(0) = \mathcal O x(0)$ \\
colums of $\mathcal O$ linearly independent. Nec \& suf: rank$\mathcal(O) = n$\\
Detectability: if all unobservable modes are stable:\\
if rank$([A^T-\lambda_i \vert C]) = n \ \forall \lambda_i \in \Lambda_A^+ \Rightarrow (A,C)$ is stabilizable.


\subsubsection{Unconstrained finite horizon optimal control problem}
Inputs $\mathbf u := [u_0^T, ..., u_{N-1}^T]^T$ minimize objective function:\\
$ J_0^*(x(0)) := \min\limits_{\mathbf u} J_0(x(0),\mathbf u) = \min\limits_{\mathbf u} x_N^T P x_N + \sum_{k=0}^{N-1} [x_k^T Q x_k + u_k^T R u_k]$\\
\hspace*{10mm}s.t. $x_{k+1} = A x_k + B u_k, k = 0,...,N-1$\\
\hspace*{17mm}$x_0 = x(0)$\\
$P \geq 0, P = P^T$ terminal weight\\
$Q \geq 0, Q = Q^T$ state weight\\
$R > 0, R = R^T$ input weight


\underline{Batch approach}: all future states represented in\\ 
terms of initial condition $x_0$ and inputs $u_0, ..., u_{N-1}$\\
$\begin{bmatrix}
 x_0 \\
 x_1  \\
 \vdots\\
 \vdots\\
 x_N
\end{bmatrix}
=
\begin{bmatrix}
 I \\
 A  \\
 A^2\\
 \vdots\\
 A^N
\end{bmatrix}
x(0)+
\begin{bmatrix}
 0 & \dots & \dots & 0\\
 B & 0 & \dots & 0 \\
 AB & B & \dots & 0\\
 \vdots & \ddots & \ddots & 0\\
 A^{N-1}B & \dots & AB & B
\end{bmatrix}
\begin{bmatrix}
 u_0 \\
 u_1  \\
 \vdots\\
 \vdots\\
 u_{N-1}
\end{bmatrix}
$

$ \mathbf x := \bar A x(0) + \bar B \mathbf u $\\
Finite horizon cost function:\\
$ J_0(x(0),\mathbf u) = \mathbf x^T \bar Q \mathbf x + \mathbf u^T \bar R \mathbf u$\\
$ \bar Q := $ blockdiag$(Q, ...,Q,P)$ and $\bar R :=$ blockdiag$(R,...,R)$\\
Eliminating $\mathbf x$ form $J_0$ gives\\
$J_0(x(0),\mathbf u) = \mathbf u^T H \mathbf u + 2 x(0) F \mathbf u + x(0)^T \bar A^T \bar Q \bar A x(0)$\\
where $H := \bar B^T \bar Q \bar B + \bar R$ and $F := \bar A^T \bar Q \bar A$. \\
Solution by setting gradient to 0: $\nabla_{\mathbf u} J_0(x(0), \mathbf u) = 0$ \\
$\mathbf u^*(x(0)) = - H^{-1}F^T x(0)$

\underline{Recursive approach}: j-step optimal cost-to-go: \\
$ J_j^*(x(j)) := \min\limits_{u_j,...,u_{N-1}} x_N^T P x_N + \sum_{k=j}^{N-1} [x_k^T Q x_k + u_k^T R u_k]$\\
\hspace*{10mm}s.t. $x_{k+1} = A x_k + B u_k, k = j,...,N-1$\\
\hspace*{17mm}$x_j = x(j)$\\
Solution by substituting equ. constr. in objective function \\and setting gradient of input to 0\\
Optimal solution for time step $k$\\
$u^*(k) = -(B^T P_{k+1} B + R)^{-1} B^T P_{k+1} A x(k) =: F_k x(k) $ \\
$P_k$ by recursion from $P_N=P$ (Discrete Time Riccati Equ):\\
$P_k = A^T P_{k+1} A + Q - A^T P_{k+1} B(B^T P_{k+1} B + R)^{-1} B^T P_{k+1} A$\\
Optimal cost-to-go: $J_k^*(x(0)) = x(k)^T P_k x(k)$

Comparison: \\
Batch approach: sequence of numerical values\\
Recursive: dynamic programming, feedback policies $u^*(k) = F_k x(k)$


\subsubsection{Infinite Horizon Control Problem}
RDE converges to constant $P$: $P_k = P_{k+1} = P_\infty $, RDE becomes ARE.\\
Feedback matrix $F_\infty$: asymptotic form of LQR\\
If $(A,B)$ stabilizably and $(Q^{1/2}, A)$ detectable, RDE converges to ARE.\\
Closed-loop system is asymptotically stable with $u(k) = F_\infty x(k)$\\\
Prove by examining cl system $x(k+1) = (A+B F_\infty)x(k)$


\textbf{State Estimation}\\
Estimate previous/current/future: smoothing/filtering/prediction\\
Model $x(k+1) = A x(k) + B u(k) + \varepsilon_1(k)$; $y(k) = C x(k) + \varepsilon_2(k)$\\
$\varepsilon_1$: Process noise; $\varepsilon_2$: Meas. noise. Zero mean: E$\{\varepsilon_i(k) \}=0$.\\
White noise has zero mean. $R_1$ big: trus meas; $R_2$ big: meas noisy.


Prediction step: $\hat x_{k | k-1} = A\hat x_{k-1 | k-1} + B u(k-1)$\\
Update step: $\hat x_{k | k} = \hat x_{k | k-1} + K (y(k)-C\hat x_{k | k-1})$\\
State estimation error $ \hat x_{i | j}^e:= x(i) - \hat x_{i | j}$\\
Estimation error dynamics, only depending upon old estimates:\\
$ \hat x_{k | k}^e:= (A-K C A) \hat x_{k-1 | k-1}^e + (I-K C)\varepsilon_1(k-1) - K \varepsilon_2(k)$\\
Stability: iff eigenvalues of $ A- K C A$ strictly inside unit circle. \\
Possible if (CA,A) is observable.
 
\textbf{Kalman Filter}

1. Compute the a priori estimate and - error covariance matrix\\
$\hat x(k|k-1) = A \hat x(k-1|k-1) + B u(k-1)$\\
$P(k|k-1) = A P(k-1|k-1) A^T +R_i$\\
2. Measure $y(k)$\\
3. Compute gain, new estimate and error covariance matrix\\
$K(k) = P(k|k-1) C^T(C P(k|k-1)C^T + R_2)^{-1}$\\
$\hat x(k|k) = \hat x(k|k-1) + K(k)(y(k)-C \hat x(k|k-1))$\\
$P(k|k) = (I-K(k)C)P(k|k-1)(I-K(k)C)^T + K(k)R_2 K(k)^T$\\
\subsection{General Optimization Problem:}
\hspace*{8mm}$\min\limits_{x \in \mathcal X} f_0(x)$ \hspace*{32mm} $f_0$: Objective function\\
\hspace*{10mm}s.t. $f_i(x)\leq 0, h_i(x) = 0$ \hspace*{8mm} $\mathcal X$: Domain of OF $f_0$\\
(Strictly) feasible point:\\
 $x \in \mathcal X$, satisf. constr. (strictly), $f_i(x) (<) \leq 0$.\\
Optimal solution:
 $x^* \in \mathcal X$ such that optimal value $f_0(x^*)\leq f_o(x)$.\\
Optimal value: $p^* = \inf \limits_{x \in \mathcal X} \{f_0(x) | f_i(x) \leq 0, h_i(x) = 0\}$.\\
Minimizer: Vector $x^*$ that achieves the minimal value.

\subsubsection{Convex Optimization Problem}
Optim prob is cvx, if objective function and feasible set are cvx. $f_0, ..f_m$ are cvx, and $h_i$ are affine.
 
\subsubsection{Definition of convex set:}
$\mathcal X$ is convex $\Leftrightarrow \lambda x + (1-\lambda) y \in \mathcal X, \forall \lambda \in [0,1], \forall x,y \in \mathcal X$


Convex Sets\\
- \underline{Subspace} $\mathcal X = \{ x \in \mathbb R^n | Ax = 0\}$. Proof: $ \forall \lambda \in [0,1], \forall x,y \in \mathcal X$ \\
 $ A(\lambda a + (1-\lambda) b) = \lambda A a + (1-\lambda)Ab = \lambda \cdot 0 + (1-\lambda) \cdot 0 = 0 $\\
- \underline{Affine space} $\mathcal X = \{ x \in \mathbb R^n | Ax = b\}$. Lines and planes.\\
- \underline{Hyperplane} $\mathcal X = \{ x \in \mathbb R^n | a^T x = b\}$. $ a \ne 0, a\in \mathbb R^n$ is the normal vector.\\
- \underline{Halfspace} $\mathcal X = \{ x \in \mathbb R^n | a^T x \leq b\}, a \ne 0,$. Can be \textbf{open ($<$)} or \textbf{closed ($\leq$)}.\\
- \underline{Set} $\mathcal X$ is a cone if $\forall x \in \mathcal X,$ and $\forall \theta > 0, \theta x \in \mathcal X$. $\theta$: scaling factor. Pointed if it contains $x=0$\\
- \underline{Conic combination of two points} $x_1$ and $x_2$: any point fulfilling $y = \theta_1 x_1 + \theta_2 x_2$\\
 for some $\theta_1, \theta_2 >0$; Convex if convex cone.\\
- \underline{Polyhedron}: intersection of finite number of closed halfspaces: $\mathcal X = \{x | a_1^T x \leq b_1,...,a_m^T x \leq b_m\} = \{x | A x \leq b\}$\\
- \underline{Polytope}: bounded polyhedron. A polyhedron is always cvx.\\
- \underline{Norm}: function $f:\mathbb R^n \rightarrow \mathbb R$ satisfying\\
 1.) $f(0) \geq 0$ and $f(x) = 0 \Rightarrow x = 0$.\\
 2.) $f(tx) = |t|f(x), t$ scalar\\
 3.) $f(x+y)\leq f(x) + f(y), \forall x, y \in \mathbb R^n$\\
 \hspace*{8mm}$l_p$ norm $\|x\|_p := [\sum_{i=1}^n |x_i|^p]^{1/p}$\\
  \hspace*{5mm}$p=1:$ Sum of abs values;\\
  \hspace*{5mm}$p=\infty:$ largest abs value $\|x\|_\infty:=max_i|x_i|$\\
- \underline{Norm ball}: $\{x | \| x-x_c\| \leq r\}, r \geq 0.$\\
- \underline{Ellipsoid}: $\{x | (x-x_c)^T A^{-1} (x-x_c)\leq 1\}, x_c$ center, $A>0$.\\
- \underline{Euclidean ball}: $\{x | \| x-x_c\|_2 \leq r\}$. (Ellipse with $A = r^2 I)$.


\underline{-Interrsection} $\mathcal X \bigcap \mathcal Y$ of two convex sets is convex:


For any $\lambda \in [0,1], \lambda a + (1-\lambda)b$ is in both $\mathcal X$ and $ \mathcal Y$. 

Therefore $\lambda a + (1-\lambda)b \in \mathcal X \bigcap \mathcal Y,  \forall \lambda \in [0,1]$ 

Set of points $ C \triangleq = \{x| ... \forall y \in Q\}: C = \bigcap_{y \in Q} C_y$

- \underline{Convex hull}: set of all convex combinations of points in $\mathcal X$.\\
- \underline{Union} $\mathcal X \bigcup \mathcal Y$ is not convex in general.


\subsubsection{Definition of convex function:}
A function $f :$ dom$(f) \rightarrow \mathbb R$ is convex iff its domain dom($f$) is cvx and
$ f( \lambda x + (1-\lambda) y) \leq \lambda f(x) + (1-\lambda) f(y)$\\
\hspace*{8mm}$  \forall \lambda \in [0,1], \forall x,y \in $ dom$(f),$\\
- $f$ is concave iff $-f$ is convex.

\underline{1st-order condition for convexity}

Differentiable function $f:$ dom$(f) \rightarrow \mathbb R$ with cvx domain is cvx iff $f(y)\geq f(x) + \nabla f(x)^T (y-x), \forall x,y \in $ dom $(f)$


2nd-order condition for convexity

Function $f :$ dom$(f) \rightarrow \mathbb R$ is cvx iff dom f$(f)$ is cvx and

Hessian $\frac{\partial^2 f(x)}{\partial x_i \partial x_j} =\nabla^2 f(x) \geq 0, \forall x \in $ dom$(f)$

- Epigraph: epi$(f) = \{ [x \, t]^T]|x \in$ dom$(f), f(x)\leq t\} \subseteq $ dom$(f) \times \mathbb R$

A function s convex iff its epigraph is cvx\\
- SubLevel set L$_\alpha = \{x | x \in $ dom$(f), f(x) \leq \alpha \}$

 $f$ is cvx $\Rightarrow$ sublevel sets of $f$ are cvx $\forall \alpha$. But not $\Leftarrow$!

 $f$ is quasi-cvs iff dom($f$) is cvx and all sublevel sets of $f$ are cvx.

\underline{Convexity-preserving operations}\\
- Non-negative weighted sum: $f$ is cvx $\Rightarrow \alpha f, \forall \alpha \geq 0$ is cvx.\\
- Composition with affine function:\\ $f$ is cvx $\Rightarrow f(Ax + b)$ is cvx.\\
- Pointwise (Pw) max:\\ If $f_1, .., f_m$ are cvx, $\Rightarrow f(x) = $ max$\{f_1, .., f_m\}$ is cvx.\\
- Pw sup: If $f(x,y)$ is cvx in $x, \forall y$, then $g(x) = $ sup$_{y \in \mathcal Y} f(x,y)$ is cvx.\\
- Min: If $f(x,y)$ is cvx in $(x,y)$ and $ \mathcal C$ is cvx, then $g(x) = \min \limits_{y \in \mathcal C} f(x,y) $ is cvx.

\subsubsection{Optimality Criterion for Differentiable $f_0$}
For cvx problem, $x$ is optimal iff it is feasible and

$\nabla f_0(x)^T (y-x) \geq 0, \forall$ feasible $y$ \\
 $a^T b = |a| |b| \cos \sphericalangle(a,b)$

Angle between gradient and any vector in the set is $< 90^\circ$.


\underline{Equivalent Optimization Problems:}

Intoducing equality constraints\\
$\min\limits_{x} f_0(A_0 x+ b_0)$ \hspace*{20mm} $\min\limits_{x} f_0(y_0)$\hspace*{10mm}$ i = 0,1,..m$\\
s.t. $f_i(A_i x+ b_i)\leq 0, i = 1,..,m$ \hspace*{3mm} s.t.$f_i(y_i)\leq 0, A_i x+ b_i = y_i$

Intoducing slack variables\\
$\min\limits_{x} f_0(x)$ \hspace*{30mm} $\min\limits_{x} f_0(y_0)$\hspace*{10mm}$ i = 1,..m$\\
s.t. $A_i x\leq b_i, i = 1,..,m$ \hspace*{10mm} s.t.$ A_i x+ s_i = b_i, \, s_i \geq 0$


\underline{Example Linear Programs}\\
-Constrained $l_\infty$ minimization:

$\min\limits_{x} \|A x - b\|\infty$ \hspace*{4mm} $\min\limits_{x,t} t$

s.t. $F x \leq g$ \hspace*{10mm} s.t. $A x - b \leq t \cdot \mathbf 1, A x - b \geq -t \cdot\mathbf  1, F x \leq g$\\
-Constrained $l_1$ minimization:

$\min\limits_{x} \|A x - b\|_1$ \hspace*{4mm} $\min\limits_{x,y} \mathbf 1^T y$

s.t. $F x \leq g$ \hspace*{10mm} s.t. $A x - b \leq y, A x - b \geq -y , F x \leq g$


\textbf{Lagrangian Function:} $L:\mathcal X \times \mathbb R^m \times \mathbb R^p \rightarrow \mathbb R$


$L(x,\lambda, \nu) = f_o(x) + \sum_{i=1}^m \lambda_i f_i(x) + \sum_{i=1}^p \nu_i h_i(x)$\\
\underline{Lagrange Dual function} $g: \mathbb R^m \times \mathbb R^p$

$g(\lambda, \nu) = \inf \limits_{x \in \mathcal X} L(x, \lambda, \nu)$

$g(\lambda, \nu)$ pointwise infimum of affine functions.

dual generates lower bounds for $p^*: g(\lambda, \nu) \leq p^*, \forall(\lambda \geq 0, \nu \in \mathbb R^p)$.\\
Procedure:
1.) Compute Lagrangian $L(x,\lambda, \nu)$

2.) Minimize it by setting gradient to 0: $\nabla L(x,\lambda, \nu)=0 \Rightarrow x^*$

3.) Substitute $x$ back into $L$ to get dual function $g(\lambda, \nu)$

4.) Lower bound property $p^*: g(\lambda, \nu) \leq p^*, \forall(\lambda \geq 0$\\
\underline{Dual Problem:} maximize dual function:

$\max \limits_{\lambda, \nu} g(\lambda, \nu)$ \hspace*{10mm} s.t. $\lambda \geq 0$\\
- Properties: Dual prob is convex, even if primal is not\\
- Optimal value $d^* \leq p^*$\\
- Point $(\lambda, \nu)$ is dual feasible if $\lambda \geq 0$ and $(\lambda, \nu) \in$ dom $g$



Weak duality: it is always true that $d^* \leq p^*$;\\ duality gap: $ d^* - p^*$\\
\underline{Strong duality:} $d^* = p^*$ often for cvx problems, check by\\
Slater condition: if there exists one strictly feasible point inside:

$\{x|Ax =b, f_i(x)<0, \forall i \in \{1,...,m\}\} \ne \emptyset$\\
If strong duality holds: $d^* = p^* \Rightarrow g(\lambda^*,\nu^*) = f_0(x^*)$


\subsubsection{Karush-Kuhn-Tucker Conditions:}

1.) Primal feasibility \hspace*{20mm}  $f_i(x^*) \leq 0, h_i(x^*) = 0$

2.) Dual feasibility \hspace*{22mm} $\lambda^* \geq 0$

3.) Complementary Slackness \hspace*{5mm} $\lambda_i^* f_i(x^*) = 0$ 

4.) Stationarity \hspace*{0mm}$\nabla_x L(x^*,\lambda^*, \nu^*) = \nabla_x f_o(x^*) + \sum_{i=1}^m \lambda_i^* \nabla_x f_i(x^*) + \sum_{i=1}^p  \nu_i^* \nabla_x h_i(x^*) = 0$


\subsection{MPC as QP} with constraints\\
$ E_u u \leq f_u;$ $E_x x \leq f_x$ \hspace*{5mm} $\Rightarrow$ \hspace*{5mm}
$\bar E_u  \mathbf u \leq \bar f_u;$ $\bar E_x  \mathbf x \leq \bar f_x;$\\
$\bar E_u = $ blockdiag$(E_u,...,E_u)$;  $\bar f_u =[f_u,...f_u]^T$\\
 $\bar f_x =[f_x,...f_x]^T$\\
$\bar E_x =
\begin{bmatrix}
  E_x & 0 & \dots  &0\\
 \vdots & \ddots & \ddots  & \vdots\\
 0 & \dots &  E_x & 0
\end{bmatrix}$
\hspace*{8mm}
$\left. \begin{matrix}
\text{Asymmetric, as} \\
\text{constraints on }x \text{ normally}\\
\text{defined for }k \in \{0,...,N-1\}
\end{matrix} \right . $

\underline{Vectorized MPC QP problem}\\
\hspace*{8mm}$\min\limits_{\mathbf u} \mathbf u^T [\bar R + \bar B^T \bar Q \bar B] \mathbf u + 2 \mathbf u^T \bar B^T \bar Q \bar A x_0$

\hspace*{8mm} s.t. $\binom{\bar E_u}{\bar E_x \bar B} \mathbf u \leq \binom{\bar f_u}{\bar f_x - \bar E_x \bar A x_0} $\\

\underline{MPC, LP $p = 1$:} $\min \sum (\|Qx\|_1) +(\|Rx\|_1)$, s.t...\\
Introduce vectors $(Z_x,Z_u)$ to model abs values of $(\bar Q\mathbf x, \bar R\mathbf u)$.\\
\hspace*{8mm}$\min\limits_{\mathbf u} \mathbf 1^T Z_x + \mathbf 1^T Z_u$\\
\hspace*{8mm} s.t. 
$\left. \begin{matrix}
\binom{\bar E_u}{\bar E_x \bar B} \mathbf u \leq \binom{\bar f_u}{\bar f_x - \bar E_x \bar A x(0)} \\
-Z_x \leq \bar Q ( \bar A x_0 + \bar B \mathbf u) \leq Z_x\\
-Z_u \leq \bar R \mathbf u \leq Z_u
\end{matrix} \right .$


\underline{MPC, LP $p = \infty$:}  $\min \sum (\|Qx\|_\infty) +(\|Rx\|_\infty)$, s.t...\\
Intro scalars $(Z_x,Z_u)$ to model largest abs vals of $(\bar Q\mathbf x, \bar R\mathbf u)$.\\
\hspace*{8mm}$\min\limits_{\mathbf u}  Z_x + Z_u$\\
\hspace*{8mm} s.t. 
$\left. \begin{matrix}
\binom{\bar E_u}{\bar E_x \bar B} \mathbf u \leq \binom{\bar f_u}{\bar f_x - \bar E_x \bar A x(0)} \\
-\mathbf 1 Z_x \leq \bar Q ( \bar A x_0 + \bar B \mathbf u) \leq \mathbf 1 Z_x\\
-\mathbf 1 Z_u \leq \bar R \mathbf u \leq \mathbf 1 Z_u
\end{matrix} \right .$


\underline{Linear or Quadratic}\\
Linear: - Easy to compute; Sol may not be unique; Far from origin: conservative; close to origin: discontinuity, dead-beat behaviour, jitter\\
Quadratic: - Connection to LQR; Sol unique; Far from origin: large inputs; close to origin: smooth action

\underline{Reference Tracking}\\
- Regulation: Reject disturbances around fix point\\
- Tracking: make output follow reference signal\\
- For prediction , model reference as constants: $ r(k+j) = r(k), j \geq 0$\\
Steady State computation\\
$\begin{bmatrix}
 (I-A) & -B\\
 C 0
\end{bmatrix}
\begin{pmatrix}
 x_{ss}\\
 u_{ss}
\end{pmatrix}
= 
\begin{pmatrix}
 0 \\
 r
\end{pmatrix}$

Resulting optimal control problem

$ V(x) := \min \limits_{\mathbf u} 	\sum\nolimits_{i=0}^{N-1} (x_i-x_{ss})^T Q (x_i-x_{ss}) + (u_i-u_{ss})^T R (u_i-u_{ss})$
s.t. $x_{i+1} = A x_i + B u_i, x_0 = x, x_i \in \mathbb X, u_i \in \mathbb U, \forall i \in \{0, ..., N-1\}$

\subsubsection{Stability \& Feasibility}
Stability not guaranteed. Check stab and feas a priori, derive conditions for all $Q$ and $R$ (conservative) or..\\
Constraints to ensure stability:\\
- Infinite prediction horizon: $N \rightarrow \infty$\\
- (Relaxed) Terminal state constraint: $x_N = 0$, $(x_N \in \mathbb X_N)$

Common idea: Use optimal value function $V^*(x)$ as Lyapunov function.\\
Assumption: Objective $f_0 (x)$ positive definite, radially unbounded, $f_0 (0) = 0$ \\
To work out, 1.) terminal state and control constraints hold in terminal set:
$ \mathbb X_n \subseteq \mathbb X, x \in \mathbb X_n \Rightarrow K(x) \in \mathbb U$\\
2.) Terminal set invariant with controller:\\
\hspace*{8mm}$ x \in \mathbb X_n \Rightarrow f(x,K(x)) \in \mathbb X_n$\\
 3.) the infinite horizon cost has to be bounded
by a terminal cost:
\hspace*{8mm}$\sum\nolimits_{i=N}^{\infty} l(x_i, K(x_i)) \leq \Psi(x_N)$\\
\underline{Recursive feasibility:}\\
1.) Assume feasibility at $k$ with $\mathbb = [u_0^*,...,u_{N-1}^*]$\\
2.) A feas sol at $k+1$ is $[\hat u_0,...,\hat u_{N-1}]=[u_1^*,...,u_{N-1}^*, K(x_N^*)]$\\
3.) Associated feasible state trajectory:\\ \hspace*{8mm}$[\hat x_0,...,\hat x_{N}]=[x_1^*,...,x_{N-1}^*, f(x_N^*,K(x_N^*))]$\\
4.) Shiftet states/inputs guaranteed to satisfy all constraints!


\underline{Stability proof:}\\
\hspace*{2mm}$V^*(x(k+1)) \leq ..$\\
\hspace*{8mm}$\sum\nolimits_{i=0}^{N-1} l(x_i^*, u_i^*)+ \Psi(x_N^*)$ \hspace*{5mm} Previous optimal cost\\
\hspace*{8mm}$- \Psi(x_N^*) -  l(x_0^*, u_0^*)$ \hspace*{5mm} Terms dropped by shifting\\
\hspace*{8mm}$l(x_N^*, K(x_N^*)) + \Psi(f(x_N^*, K(x_N^*)))$\hspace*{5mm} Terms added\\
It follows: $V^*(x(k+1)) \leq V^*(x(k)) - l(x_0^*, u_0^*)$\\
 (with the Lyapunov condition:)\\
\hspace*{8mm}$\Psi(f(x,K(x)))-\Psi(x) \leq -l(x,K(x)) \forall x \in \mathbb X_N$\\
This tends to zero, so cost tends to zero, so $x(k) \rightarrow 0$.


\underline{Choice of terminal weight P:} $\Psi(x) = x^T P x$\\
1.) Linear, unconstrained and stable system with quadratic cost: Controller: $K(x) = 0$, terminal constraint $\mathbb X_N = \mathbb R^n$. Determine $P$ from Lyapunov equation. \\
2.) Linear, constrained, unstable sys. with quadratic cost: Controller $K(x) = Kx$ (any stabilizing contr), terminal set $\mathbb X_n$: invariant set for $x(k+1)=(A+BK)x$.  Make $P$ equal optimal cost to go from $N$ to $\infty$ by solution of ARE. Assumes no constraints are active after $N$.\\
3.) Desire of state and input = zero at end. No $P$ but constraint $x_{k+N}=0$. 

\subsubsection{Linear Integer Inequalities} Linear Integer Inequalities can represent logical propositions\\
Idea: associate to each boolean variable $p_i$ a binary integer variable $\delta_i$:
$p_i \Leftrightarrow\{\delta_i = 1\}, \neg p_i \Leftrightarrow\{\delta_i = 0\}$

For a logic proposition $\Omega(p_i)$ it is always possible to define a set of linear inequalities:
$A \delta \leq B, \delta \in \{0,1\}^n$


\underline{ Analytic approach}

Conjunctive normal form (CNF)

$\Omega(p_i) = \bigwedge_j [\bigvee_j p_i]$

Logic proposition in CNF into algebraic inequalities

 $\neg p_i, \, \, \, 1-\delta_i,$ \hspace*{8mm} Not\\
 $ p_i \vee p_j, \, \, \, \delta_i+\delta_j \geq 1, $ \hspace*{8mm} Or\\
 $ p_i \wedge p_j, \, \, \, \delta_i+\delta_j \geq 2, $ \hspace*{8mm} And\\
 $ p_i \Rightarrow p_j, \, \, \, \delta_i-\delta_j \geq 0, $ \hspace*{8mm} if $p_i$ then $p_j$\\
 $ p_i \Leftrightarrow p_j, \, \, \, \delta_i-\delta_j = 0, $ \hspace*{8mm} logic equality
 
 Rules\\
 $ \neg (A \wedge B) = \neq A \vee \neg B$\\
 $A \wedge (B \vee C) = (A \wedge B) \vee (A\wedge C)$\\
 $ A \vee (B \wedge C) = (A\vee B)\wedge (A \vee C)$
 
 \underline{ Geometric approach}
 
 Idea: Polytope $\mathcal P = \{ \delta \in \{0,1\}^n | A\delta \leq B\}$ is the convex hull of the rows of the truth table of proposition $\Omega(p_i)$.


\underline{Combining logic rules and continuous dynamics}

e.g. $p \Leftrightarrow a^Tx \leq b, \mathcal X = \{x | a^T x-b \in [m, M]\}$.\\
 Bounds $m, M$ have to be specified. Translated to linear inequalities:\\
 \hspace*{8mm}$ a^T x-b \leq M(1-\delta)$\\
 \hspace*{8mm}$ a^T x-b > m \delta$
 
 \textbf{MPC for Hybrid Systems} Quadratic norm constrained finite time optimal control problem

$ J^*(x_t) = \min \limits_U \|x_{t+N}\|_P + \sum_{k=0}^{N-1} \|x_{t+k}\|_{Qx}^2 +\|u_{t+k}\|_{Qu}^2+\|\delta_{t+k}\|_{Q\delta}^2+\|z_{t+k}\|_{Qz}^2,$\\
\hspace*{8mm} $ s.t. \left \{  \begin{matrix}
 x_{t+k+1} = A x_{t+k} + B_1 u_{t+k} + B_2 \delta_{t+k} + B_3 z_{t+k}\\
 E_2 \delta_{t+k} + E_3 z_{t+k} \leq E_4 x_{t+k} + E_1 u_{t+k} + E_5\\
 x_{t+N} \in \mathcal X
\end{matrix} \right . $

\underline{Mixed Logical Dynamical (MLD) Model}\\
see optimization problem before, with $k = 0$.

\underline{Piecewise Affine (PWA) System}\\
- polyhedral partition of the $(x, u)$-space:\\
 \hspace*{8mm}$ \{\mathcal D^i\}_{i=1}^D := \{\tbinom{x_t}{u_t} | P_x^i x_t + + P_u^i u_t \leq P_c^i\}$\\
- affine dynamics in each region:\\
\hspace*{8mm} $\left \{  \begin{matrix}
x_{t+1} = A^i x_t + B^i u_t + f^i\\
y_t = C^i x_t + D^i u_t + g^i
\end{matrix} \right \}
$ if $ x_t \in \mathcal D^i$

- any well-posed (for a given $[x_t^T u_t^T]^T \Rightarrow x_{t+1}, y_t$ are uniquely determined) PWA system can be represented by an MLD system, assuming that the set of feasible states and inputs is bounded.\\
- a completely well-posed (well-posed + uniquely determined $\delta_t, z_t \forall [x_t^T u_t^T]^T$) MLD can be written as a PWA system.


\subsection{Numerical Optimization Methods}
\subsubsection{Unconstrained Optimization}
\textbf{Algorithm performance Measurements}
$\bullet$ Convergence: is \textit{m} finite for every $\epsilon, \delta > 0$\\
$\bullet$ Convergence speed: dependence of errors $f(x_m) - f(x^*)$ and $dist(x_m, \mathbb{Q})$\\
$\bullet$ Feasability: For some methods $\delta=0$, but in general $\delta \neq 0$\\
$\bullet$ Numerical robustness: Robustness in presence of finite presicion arithmetics\\
$\bullet$ Warm starting: Improve performance by initializing Algorithm $x_0$ near $x^*$\\
$\bullet$ Preconditioning: Transform problem $P$ into transformed Problem $\tilde{P}$


\textbf{Gradient Method}
\begin{empheq}[innerbox=\fbox]{align*}
Set & \hspace{0.5cm} x_0\\
Repeat & \hspace{0.5cm} x_{i+1} = x_i - \frac{1}{L} \cdot \nabla f(x_i)\\
until & \hspace{0.5cm} f(x^m) - f(x^*) \leq \epsilon_1 or \Vert x_m - x_{m-1} \Vert \leq \epsilon_2
\end{empheq}
\hspace{0.5cm} Convergence $m \sim \mathcal O(L\|x_o - x^*\|^2/\epsilon_1)$\\
Assumptions:\\
$\bullet$ $\nabla f$ is Lipschitz-continous with Lipschitz constant $L$:\\
\hspace*{0.5cm}  $\Vert \nabla f(x) - f(y) \Vert \leq L \Vert x-y \Vert \forall x,y \in \mathbb{R}^n$\\
$\bullet$ $f$ can be upper bounded by a quadractic function:\\
\hspace*{0.5cm}$f(x) \leq f(y) + \nabla f(y)^T (x-y) + \frac{L}{2} \Vert x-y \Vert^2 = \bar f(x,y), \forall x,y $\\
Convergence:\\
\hspace*{0.5cm}  $f(x_{i-1}) \leq \bar f(x_{i-1}, x_i) \leq \bar f(x_i,x_i) = f(x_i)$\\
\underline{Strong convexity:} $f(x)$ upper and lower bounded\\
 \hspace*{0.5cm} $ \bar f(x,y)|_{L=\mu} \leq f(x) \leq \bar f(x,y)|_{L=L}$\\
For $f(x)=x^T H x$: $L = \max(\text{eig}(H))$, $\mu = \min(\text{eig}(H)$.\\
Condition number: $\kappa \triangleq = \frac{L}{\mu}$ .\\
\hspace*{0.5cm}  problem badly conditioned (slow convergence):  $\kappa  \geq 1$

\textbf{Fast Gradient Method}
\begin{empheq}[innerbox=\fbox]{align*}
Set & \hspace{0.5cm} x_0, y_0 = x_0  \text{ and }  \alpha_0 = (\sqrt{5}-1)/2\\
Repeat & \hspace{0.5cm} x_{i+1} = y_i - \frac{1}{L} \cdot \nabla f(x_i)\\
			& \hspace{0.5cm} \alpha_{i+1} = \alpha_i (\sqrt{\alpha_i^2+4} - \alpha_i)/2\\
			& \hspace{0.5cm} \beta_i = \frac{\alpha_i(1-\alpha_i)}{\alpha_i^2+\alpha_{i+1}}\\
			& \hspace{0.5cm} y_{i+1} = x_{i+1} + \beta_i (x_{i+1} - x_i)  \text{ for } i = 0, ..., m\\
until & \hspace{0.5cm} f(x^m) - f(x^*) \leq \epsilon_1 or \Vert x_m - x_{m-1} \Vert \leq \epsilon_2
\end{empheq}
\hspace{0.5cm} Convergence $m \sim \mathcal O(\sqrt{L\|x_o - x^*\|^2/\epsilon_1)}$

\textbf{Newtons Method:}\\
Min of 2nd-order approx of $f$ at point $x_i$\\
$x_{i+1} = \min f(x_i) + \nabla f(x_i)^T v + 1/2v^T \nabla^2 f(x_i)v; v = (x-x_i)$\\
\hspace*{0.5cm} $x_{i+1} = x_i + h_i \Delta x_{nt}$\\
Newton direction: $\Delta x_{nt} = -(\nabla^2f(x_i))^{-1} \nabla f(x_i)$\\
Finding $h_i$ is not as easy as in other cases:\\
Exact way: Compute optimal $h_i$: $h_i^* = \underset{h > 0}{\operatorname{argmin}} f(x_i + h_i \Delta x_{nt})$\\
Inexact: Find $h_i$ that decreases $f$ by some $\%$, for example:\\
 Backtracking: $\alpha \in (0, 0.05)$, $\beta \in (0,1)$\\
\textbf{while} $f(x_i + h_i \Delta x_{nt}) > f(x_i) + \alpha h_i \nabla f(x_i)^T \Delta x_{nt}$ \textbf{do} $h_i \leftarrow \beta h_i$

\subsubsection{Constrained Optimization}
Constraint GM = GM + projection $\pi_{\mathbb{Q}}(.)$ on feasible set. \\
\textbf{Gradient method}: $x_{i+1} = \pi_{\mathbb{Q}}(x_i - \frac{1}{L} \cdot \nabla f(x_i))$\\
\textbf{Fast gradient method} with: $x_{i+1} = \pi_{\mathbb{Q}}(y_i - \frac{1}{L} \cdot \nabla f(x_i))$\\
- If $\pi_{\mathbb{Q}}(.)$ easy to compute: very fast algo.\\
- If $\pi_{\mathbb{Q}}(.)$ only computed numerically: solve also dual -> slower. \\
Cheap sets: hyperplane, affine set, halfspace, 1\&2-norm ball...\\
\textbf{Interior Point Method}
Unconstrained problem, where constraints are moved to objective via indicator functions $\Phi$.\\
$\min \hspace{0.1cm} f(x) \hspace{2cm} \Leftrightarrow \hspace{1cm} x^*(\kappa) = \min f(x) + \kappa \Phi (x)$\\
s.t. $g_i(x) \leq 0, i=1,...,m$\\
With the \textbf{barrier function}: $\Phi(x) = - \sum\limits_{i=1}^m \log (-g_i(x))$

Central path: $\{x^*(\kappa)|\kappa>0\}$: optimal solution for all $\kappa > 0$.

Approximation improves as $\kappa \rightarrow 0$

\underline{Barrier interior-point method:} follow central path to optimal solution .

Start with: strictly feasible $x_0, \kappa_0, \mu>1$\\
repeat \hspace*{5mm}1. Centering step: compute $x^*(\kappa_i)$\\
	\hspace*{15mm}2. update $x_i = x^*(\kappa_i)$\\
	\hspace*{15mm}3. stop if $m \kappa_i<0$ ( duality gap)\\
	\hspace*{15mm}4. Decrease barrier parameter : $\kappa_{i+1} = \kappa_i/\mu$
%Typically $\kappa$ is first chosen big an then itteratively smaller, whereby the barrier function increasingly gets smaller. By using the found $x^*$ of the previous itteration as a warm start, the solution converges to the actual $x^*$.


\subsection{Parametric Programming}
Parametric programming is a way to evade solving a MPC programm online, but rather moving the computational effort offline. For this approach, the feasible set is divided in subsets for which the same constraints are active: \textbf{Critical Regions}:
Subset of paramet set, where local optimality conditions (KKT) do not change (same constraints are active).\\
\hspace*{5mm}$ \theta \in \mathcal X \in \mathbb R^n$: vector of parameters.


\subsubsection{Multi-parametric Linear Problem (mpLP)}
For linear problems of the following form. \\
\textbf{Primal Problem}:\\
$J^*(\theta) = \min\limits_{z} c^T z$ \hspace{1cm} s.t. $Gz \leq W+S \theta$\\
\textbf{Dual Problem}\\
$\max\limits_{\pi} (W + S \theta)^T \pi$ \hspace{1cm} s.t. $G^T \pi = c, \hspace{1cm} \pi \leq 0$\\
\underline{Active} and \underline{inactive} sets are defined as follows:\\
Active indices: $\mathcal{A}(\theta) = \{ i \in \mathcal{I} | \forall z : J(z, \theta) = J^*(\theta) \Rightarrow G_i z - S_i \theta - W_i = 0 \}$\\
Inactive indices: $\mathcal{N}(\theta) = \{ i \in \mathcal{I} | \forall z : J(z, \theta) = J^*(\theta) \wedge G_i z - S_i \theta - W_i < 0 \}$\\
The algorithm for finding the regions contains 3 steps:\\
\textbf{1. Step}\\
For an inital vector $\theta_0$ solve Primal and Dual problem: Find $z^*$ and $\pi^*$\\
Obtain sets of inactive and active constraints:\\
Obtain sets $\mathcal{A}(\theta_0)$ and $\mathcal{N}(\theta_0)$\\
$G_{\mathcal{A}}, S_{\mathcal{A}}, W_{\mathcal{A}}=\{G_i, S_i, W_i | i \in {\mathcal{A}} \}$\\
$G_{\mathcal{N}}, S_{\mathcal{N}}, W_{\mathcal{N}}=\{G_i, S_i, W_i | i \in {\mathcal{N}} \}$\\
\textbf{2. Step}\\
Compute optimizer: $z^*(\theta) = G_{\mathcal{A}}^{-1} S_{\mathcal{A}} \theta + G_{\mathcal{A}}^{-1} W_{\mathcal{A}} = F_0 \theta + g_0$\\
Critical Region: $\mathcal{CR}_0 = \{\theta | (G_{\mathcal{N}} F_0 - S_{\mathcal{N}}) < W_{\mathcal{N}} - G_{\mathcal{N}} \cdot g_0$\\
\textbf{3. Step}\\
Explore rest of ${\mathcal{X}}$ for example by reversing inequalities (may yield overlapping or artificially splitted Regions) or by just going to a boundary and crossing it by going in direction of gradient by some (empirical) delta.


\subsubsection{Multi Parametric Quadratic Problem mpQP}
\textbf{Primal Problem}:\\
$J^*(\theta) = \min\limits_{z} \hspace{0.25cm} 1/2 \cdot z^T H z$ \hspace{1cm} s.t. $Gz \leq W+S \theta$\\
\textbf{1. Step}\\
$\bullet$ Solve QP for one $\theta_0$, obtain optimizer $z*(\theta)$\\
$\bullet$ Identify indices of active $\mathcal{A}(\theta)$ and inactive $\mathcal{N}(\theta)$ sets and corresponding matrices $\{G_i, S_i, W_i\}_\mathcal{A}$ and $\{G_i, S_i, W_i\}_\mathcal{N}$.\\
\textbf{2. Step}\\
Compute $z^*(\theta)$, $\lambda^*(\theta)$ and $\mathcal{CR}_0$:\\
$z^*(\theta) = H^{-1} G_{\mathcal{A}}^T (G_{\mathcal{A}} H ^{-1} G_{\mathcal{A}}^T)^{-1}(W_{\mathcal{A}} + S_{\mathcal{A}} \theta) $\\
$\lambda^*(\theta) = -(G_{\mathcal{A}} H^{-1} G_{\mathcal{A}}^T)^{-1}(W_{\mathcal{A}} + S_{\mathcal{A}} \theta) $\\
$\mathcal{CR}_0 = \{\theta | A \theta \leq b\}$\\
%This is the A matrix and b matrix of the mpQP program
$A = \begin{bmatrix} 
G H^{-1} G_{\mathcal{A}}^T (G_{\mathcal{A}} H ^{-1} G_{\mathcal{A}}^T)^{-1} S_{\mathcal{A}} - S\\ 
(G_{\mathcal{A}} H^{-1} G_{\mathcal{A}}^T)^{-1} S_{\mathcal{A}}
\end{bmatrix}\\
% b matrix
b = 
\begin{bmatrix}
W + G H^{-1} G_{\mathcal{A}}^T (G_{\mathcal{A}} H ^{-1} G_{\mathcal{A}}^T)^{-1} W_{\mathcal{A}}\\
(G_{\mathcal{A}} H^{-1} G_{\mathcal{A}}^T)^{-1} W_{\mathcal{A}}
\end{bmatrix}
$

\textbf{3. Step}\\
Explore Rest of $\mathcal{X}$ analogously to mpLP.


 
\subsection{Calculation Rules:}
\begin{align*}
(A \cdot b)^T &= b^T \cdot A^T\\
(A \cdot b)^{-1} &= b^{-1} \cdot A^{-1}\\
(A^{-1})^T &= (A^T)^{-1}\\
\nabla_x (x^T \cdot b) &= \nabla_x (b^T \cdot x) = b\\
\nabla_x (b \cdot x) &= b^T\\
\nabla_x (x^T \cdot Q \cdot x) &= 2 \cdot Q \cdot x, \hspace{0.5cm} Q \succeq 0\\
x^T \cdot A^T \cdot Q \cdot A \cdot x &= (x^T \cdot A^T \cdot Q \cdot A \cdot x)^T,  \hspace{0.5cm} Q \succeq 0\\
Q & = Q^T, \hspace{0.5cm} Q \succeq 0\\
det(A \cdot B) &= det(A) \cdot det(B)\\
det(A^T) &= det(A)\\
det(A^{-1}) &= 1/det(A)\\
\{-1,1\}: & \text{ either -1, or 1}\\
[1, 2[: & \text{ values from including 1 to excluding 2}
\end{align*}




\end{multicols*}
\end{document}